\documentclass[
paper=a4,                       % paper size
fontsize=11pt,                  % font size
headinclude=false,              % header does not belong to the text
footinclude=false,              % footer does not belong to the text
pagesize,                       % set the pagesize in a DVI document
]{scrartcl}

\include{common}


\newcommand{\freshs}{\mbox{\textsf{FRESHS}}\xspace}


\begin{document}

\esptitlehead

\title{Setting up \es{} for FFS sampling with \freshs}

\author{E. Ribeiro Tzaras}

\maketitle

\section{Introduction}

Welcome to the forward flux sampling (FFS) \es{} tutorial!

This tutorial assumes basic \es{} and FFS knowledge. We will use the software package \freshs which so far has three rare event sampling techniques implemented, namely SPRES, FFS and PERMFFS.
We will focus on the direct FFS method (sometimes called dFFS in the literature) to drive a system through a transition.
General instructions for arbitrary rare event systems will be accompanied by a model system of unbiased polymertranslocation.
First we will discuss how to setup the \freshs server and client configurations, as shown in Sec.~\ref{sec:freshs_setup}.
%\es{} can be used with advanced sampling techniques such as FFS (forward flux
%sampling) with \freshs (flexible rare event sampling harness system).

\section{\freshs Setup} \label{sec:freshs_setup}


\section{Changes to your simulation script} \label{sec:simulation_script}


\section{Running your simulation} \label{sec:run_ffs}



\section{Obtaining transition rates and interface statistics} \label{sec:freshs_analysis}







\end{document}